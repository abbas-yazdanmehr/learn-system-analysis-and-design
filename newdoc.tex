%%% you can find and change template keywords by 
%%% searching "yyy" all of this template keywords 
%%% starts by "yyy".

\documentclass[a4paper, 12pt]{article}


%%% packages %%%
\usepackage[T1]{fontenc}
\usepackage{tgbonum}
\usepackage{tabto}
\usepackage[top=2cm, bottom=2cm, right=2cm, left=2cm]{geometry}
\usepackage{xcolor}
\usepackage{proof}
\usepackage{amsmath}
\usepackage{amssymb}
\usepackage{tabularx}
\usepackage{ulem}
\usepackage{listings}
\usepackage{hyperref}
\usepackage{graphicx}
\usepackage[english]{babel}
\usepackage{amsthm}


%%% defines %%%
\def\pfbox % new experimental version (DEK, November 88)
{{\ooalign{\hfil\lower.06ex % a smiley face
\hbox{$\scriptscriptstyle\frown$}\hfil\crcr
 \hfil\lower.7ex\hbox{\"{}}\hfil\crcr
 \mathhexbox20D}}}

%%% newcounter %%%
\newcounter{stepcounter}
 
%%% newenvironment %%%


%%% new theoremm %%%
\theoremstyle{definition}
\newtheorem{definition}{Definition}[section]
\newtheorem{theorem}{Note}[section]

%%% definecolors %%%
\definecolor{dkgreen}{rgb}{0,0.6,0}
\definecolor{gray}{rgb}{0.5,0.5,0.5}
\definecolor{mauve}{rgb}{0.58,0,0.82}
\definecolor{backcolour}{rgb}{0.95,0.95,0.92}
\definecolor{blue1}{rgb}{0, 0, 0.55}
\definecolor{green1}{rgb}{0, 0.4, 0}
\definecolor{red1}{rgb}{0.55, 0, 0}
\definecolor{purple1}{rgb}{0.3, 0, 0.5}

%%% newcommand %%%
\newcommand{\noticon}{\includegraphics[width=15pt]{constants/note.png}}
\newcommand{\deficon}{\includegraphics[width=20pt]{constants/def.png}}
\newcommand{\piticon}{\includegraphics[width=25pt]{constants/pitfall.png}}
\newcommand{\stepicon}{\includegraphics[width=25pt]{constants/step.png}}
\newcommand{\cmdicon}{\includegraphics[width=20pt]{constants/terminal.png}}

\newcommand{\mynote}[1]{\vspace{10pt} 
\noindent
  \hspace{0.5em}\raisebox{-0.2em} \noticon \quad \textbf{\textcolor{blue1}{#1}} 
}

\newcommand{\mydef}[1]{\vspace{10pt} 
\noindent
  \hspace{0.5em}\raisebox{-0.3em} \deficon \quad \textit{\textbf{\textcolor{blue1}{#1}}} \quad
}

\newcommand{\mypit}[1]{\vspace{10pt} 
\noindent
  \hspace{0.5em}\raisebox{-0.4em} \piticon \quad \textbf{\textcolor{red1}{#1!}} 
\\[5pt]}

\newcommand{\mystepbystep}[1]{\setcounter{stepcounter}{1}
\vspace{10pt} 
\noindent
  \hspace{0.5em}\raisebox{-0.5em} \stepicon \quad {\large \textbf{\textcolor{blue1}{#1}}}
  \\[5pt]
}

\newcommand{\mystep}[1]{
    \noindent
    \begin{tabular}{p{0.1\textwidth} p{0.85\textwidth}}
      {\Large \thestepcounter}& #1
    \end{tabular} 
    \addtocounter{stepcounter}{1} \\[0.5cm]
}

\newcommand{\mycommand}[1]{
\vspace{10pt} 
\noindent
  \hspace{0.5em}\raisebox{-0.5em} \cmdicon \quad {\textbf{\textcolor{blue1}{#1}}}
}

\newcommand{\mycodeinput}[1]{\noindent \lstinputlisting[firstline=0, lastline=100, caption=#1, captionpos=b]{#1}}

%%% package settings %%%
\lstset{
  backgroundcolor=\color{backcolour},
  frame=tblr,
  % language=none,
  aboveskip=3mm,
  belowskip=3mm,
  showstringspaces=false,
  columns=flexible,
  basicstyle={\small\ttfamily},
  numbers=none,
  numberstyle=\tiny\color{gray},
  keywordstyle=\color{blue},
  commentstyle=\color{dkgreen},
  stringstyle=\color{mauve},
  breaklines=true,
  breakatwhitespace=true,
  tabsize=3
}



%%% begin document %%%
\begin{document}

\title{Learn System Analysis And Design}
\author{Abbas Yazdanmehr \\ \texttt{abbas.yazdanmehr1@gmail.com}}
\date{last update: \today}
\maketitle

\newpage

\tableofcontents

\newpage

\section{Usecase Modeling (FAST development framework)}
One of the most important things in system development is finding \textcolor{red1}{Requierments}.

\begin{itemize}
  \item These Requierments should be categorized in good categories like 
\textit{Essential Requierments} and \textit{True Requierments}.
  \item These Requierments should be very clear and understandable.
  \item Then these Requierments are validable and measureable.
\end{itemize}

\mynote{Drawback of using just other Models like process, data and object Models}
\begin{itemize}
  \item Too Technical for not Technical people.
  \item Usecase Model is a base for Modeling process, data and object.
\end{itemize}

By the Usecase Model you can see all Requierments in a Tree sequence dependency.

\mynote{Usecase Model idea}
Usecase Model come from Object Oriented in Technical subjects originally.

\mynote{Feasible system development}
Every system development process in first step should be understandable, and then
it should optimize. Other system development way that don't follow these rules works
by something like pattern recognition and artificial intelligence.

\mydef{Usecase}
A sequence of steps of one narrative in system for doing \textcolor{red1}{one duty}.

\mydef{Usecase Diagram}
A Graphic Diagram for interaction between system with users and other systems.

\mydef{Usecase Narrative}
Complement(Complete Description) of Usecase Diagram.
\newpage

\mystepbystep{Drawing Usecase Diagram}

\mystep{Diagram parts Meaning:
\begin{itemize}
  \item Usecase: oval
  \item Actor: stickman
  \item System/Subsystem: rectangle
  \item Connection: lines/vectors
  \item Extended Connection: dashed line with <<extends>>
  \item Extend Conditions: paper
  \item Include/Uses Connection: dashed line with <<include>>
  \item Depends on Connection: line with <<depends on>>
  \item Inheritance Connection: vector from Actor to Actor
\end{itemize}
}
Extension Usecase is optional or conditional. It have base Usecase. \\
Include Connection have abstract Usecase that can reuse many times. \\
Depends on Connection determine sequence of Usecase. \\
Inheritance Connection have abstract Actor. \\

\mystep{Determine Diagram parts:
\begin{enumerate}
  \item Identify Actors.(User-Centered Design)
  \item Identify Usecases.
  \item Draw Usecase Diagram.
  \item Write Usecase Narrative.
\end{enumerate}
}



\end{document}